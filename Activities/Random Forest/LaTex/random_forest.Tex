\documentclass{article}
\usepackage{graphicx}
\usepackage[utf8]{inputenc}
\usepackage{float}

\title{Random Forest}
\author{Alain Lobato}
\date{30 de Marzo del 2025}

\begin{document}

\maketitle

\section{Introducción}
Un Random Forest es un algoritmo de aprendizaje automático de uso común, el cual utiliza distintos árboles de decisión para obtener un resultado único. Este surge debido al \textit{overfitting} en los árboles de decisión. Se crea este modelo para evitar este problema, memorizando soluciones en lugar de realizar un aprendizaje generalizado, trabajando distintos árboles en conjunto.

\section{Metodología}
Primero clonamos nuestro repositorio de la materia en nuestra máquina y luego creamos una carpeta llamada ``Random Forest''. Descargamos el archivo CSV para extraer los datos y creamos un archivo llamado \texttt{random\_forest.py} donde desarrollamos nuestro código usando Visual Studio Code.

Iniciamos importando las librerías necesarias, instalando las dependencias en caso de ser necesario.

\begin{figure}[H]
    \centering
    \includegraphics[width=0.8\textwidth]{img/1.png}
    \caption{Importación de librerías}
\end{figure}

Ahora cargaremos los datos e imprimimos con \texttt{head} los primeros 5 registros, así como sus dimensiones.

\begin{figure}[H]
    \centering
    \includegraphics[width=0.8\textwidth]{img/2.png}
    \caption{Carga de datos}
\end{figure}

\begin{figure}[H]
    \centering
    \includegraphics[width=0.8\textwidth]{img/3.png}
    \caption{Resultados de la carga de datos}
\end{figure}

Ahora bipartimos los datos en clases, donde 0 es un acceso válido y 1 es un acceso fraudulento:

\begin{figure}[H]
    \centering
    \includegraphics[width=0.8\textwidth]{img/4.png}
    \caption{Clasificación de datos}
\end{figure}

Podemos ver cómo se encuentra un registro más cargado que el otro, con demasiados registros válidos y pocos registros fraudulentos.

Ahora creamos el dataset, separamos los datos válidos y fraudulentos, así como hacemos una división entre los datos independientes y nuestro \textit{target}.

\begin{figure}[H]
    \centering
    \includegraphics[width=0.8\textwidth]{img/5.png}
    \caption{Separación de datos}
\end{figure}

Ahora crearemos nuestro Random Forest usando los datos que ya separamos y lo entrenamos.

\begin{figure}[H]
    \centering
    \includegraphics[width=0.8\textwidth]{img/6.png}
    \caption{Creación y entrenamiento del modelo}
\end{figure}

\section{Resultados}
Vamos a predecir nuestros datos de test y mostramos los resultados usando la función que anteriormente declaramos:

\begin{figure}[H]
    \centering
    \includegraphics[width=0.8\textwidth]{img/7.png}
    \caption{Predicción de datos de prueba}
\end{figure}

\begin{figure}[H]
    \centering
    \includegraphics[width=0.8\textwidth]{img/8.png}
    \caption{Resultados de la predicción}
\end{figure}

Podemos ver que se tienen buenos resultados. El reporte de clasificación arroja un \textit{recall} de 0.98 y un F1-score macro promedio de 0.55, siendo muy buenos resultados.

\begin{figure}[H]
    \centering
    \includegraphics[width=0.8\textwidth]{img/9.png}
    \caption{Reporte de clasificación}
\end{figure}

\section{Conclusión}
Gracias a los Random Forest, se pudo evitar el problema de \textit{overfitting} que ocurre al usar solo un árbol de decisión, lo que garantiza un mejor resultado a la hora de predecir algún dato. No fue necesario hacer ningún ajuste al código y todo resultó sencillo de realizar. Se repasó muy bien la información en esta práctica.

\section{Referencias}
Material de clase (2025). UANL.\\
Bagnato J. (2010). \textit{Aprende Machine Learning}. Leanpub.

\end{document}
