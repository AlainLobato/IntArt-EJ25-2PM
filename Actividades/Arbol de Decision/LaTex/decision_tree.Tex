\documentclass[a4paper,12pt]{article}
\usepackage[utf8]{inputenc}
\usepackage{graphicx}
\usepackage{amsmath}
\usepackage{hyperref}
\usepackage{booktabs}
\usepackage{float}

\title{\textbf{Árboles de Decisión en Machine Learning}}
\author{Alain Lobato}
\date{30 de Marzo del 2025}

\begin{document}

\maketitle

\section{Introducción}
Los árboles de decisión constituyen gráficamente la solución en posibles escenarios por condiciones, es uno de los algoritmos más utilizados en Machine Learning (ML). Inician en un nodo llamado raíz, a partir del cual se descomponen en más nodos que evalúan condiciones como ciertas o falsas. Se biparte hasta llegar a las hojas, que representan los nodos finales. Este algoritmo nos ayuda a analizar los datos para tomar la mejor decisión basada en estos datos y hacer predicciones.

\section{Metodología}
\subsection{Espacio de trabajo}
Primero clonamos nuestro repositorio de la materia en nuestra máquina, luego creamos una carpeta llamada ``Árbol de Decisión''. Descargamos el archivo CSV para extraer los datos y creamos un archivo llamado \texttt{decision\_tree.py} donde desarrollamos nuestro código usando Visual Studio Code.

Iniciamos importando las librerías necesarias, instalando las dependencias en caso de ser necesario.

\begin{figure}[H]
    \centering
    \includegraphics[width=0.8\textwidth]{1.png}
    \caption{Librerias necesarias.}
    \label{fig:libraries}
\end{figure}

\subsection{Carga de datos}
Leemos el archivo CSV de \textit{artists billboard} y verificamos su tamaño y los primeros 5 registros:

\begin{figure}[H]
    \centering
    \includegraphics[width=0.8\textwidth]{2.png}
    \caption{Lectura de archivo.}
    \label{fig:file_reading}
\end{figure}

\begin{figure}[H]
    \centering
    \includegraphics[width=0.8\textwidth]{3.png}
    \caption{Resultado en consola de los registros.}
    \label{fig:console_output}
\end{figure}

\texttt{Esto nos muestra: 635 registros en 11 características. Se visualizan las primeras 5 canciones registradas.}


Agrupamos las canciones que fueron top en Billboard, asignando 1 si lo fueron y 0 si no lo fueron:

\begin{figure}[H]
    \centering
    \includegraphics[width=0.8\textwidth]{4.png}
    \caption{Agrupamos canciones por top Billboard.}
\end{figure}

\begin{figure}[H]
    \centering
    \includegraphics[width=0.8\textwidth]{5.png}
    \caption{Impresion en consola.}
\end{figure}

\texttt{494 canciones no han sido top Billboard y 141 sí lo han sido.}

\texttt{Ahora analizaremos cada cancion por aspecto: }
\begin{figure}[H]
    \centering
    \includegraphics[width=0.8\textwidth]{6.png}
    \caption{Resultado en consola de los registros.}
\end{figure}
\begin{itemize}
    \item Tipo de artista.
    \begin{figure}[H]
        \centering
        \includegraphics[width=0.8\textwidth]{7.png}
        \caption{Resultado de analisis por tipo de artista.}

    \end{figure}
    \item Mood.
    \begin{figure}[H]
        \centering
        \includegraphics[width=0.8\textwidth]{8.png}
        \caption{Resultado de analisis por mood.}

    \end{figure}
    \item Tempo.
    \begin{figure}[H]
        \centering
        \includegraphics[width=0.8\textwidth]{9.png}
        \caption{Resultado de analisis por tempo.}

    \end{figure}
    \item Género.
    \begin{figure}[H]
        \centering
        \includegraphics[width=0.8\textwidth]{10.png}
        \caption{Resultado de analisis por género.}

    \end{figure}
    \item Año de nacimiento.
    \begin{figure}[H]
        \centering
        \includegraphics[width=0.8\textwidth]{11.png}
        \caption{Resultado de analisis por año de nacimiento.}

    \end{figure}
\end{itemize}

Ahora analizamos las canciones top y las que no lo fueron en función de las fechas \textit{chart}, obteniendo los siguientes resultados:

\begin{figure}[H]
    \centering
    \includegraphics[width=0.8\textwidth]{12.png}
    \caption{Codificacion de analisis.}
\end{figure}

\begin{figure}[H]
    \centering
    \includegraphics[width=0.8\textwidth]{13.png}
    \caption{Resultado del grafico de analisis.}
\end{figure}

\subsection{Preprocesamiento de datos}
Sustituimos los años que sean 0 por un valor \texttt{None}. Calculamos la edad restando el año de aparición menos el año de nacimiento usando una función lambda, asignándola a \texttt{edad\_en\_billboard}.

\begin{figure}[H]
    \centering
    \includegraphics[width=0.8\textwidth]{14.png}
    \caption{Codificacion para sustitucion de edades por None.}
\end{figure}

Asignamos edades aleatorias basadas en el promedio y la desviación estándar, visualizándolas en un gráfico.

\begin{figure}[H]
    \centering
    \includegraphics[width=0.8\textwidth]{15.png}
    \caption{Asignacion de edades aleatorias.}
\end{figure}

\begin{verbatim}
Edad promedio, desviación estándar e intervalo de edades a asignar.
\end{verbatim}

\begin{figure}[H]
    \centering
    \includegraphics[width=0.8\textwidth]{16.png}
    \caption{Analisis de lo que obtenemos en consola.}
\end{figure}

\begin{figure}[H]
    \centering
    \includegraphics[width=0.8\textwidth]{17.png}
    \caption{Resultado del grafico.}
\end{figure}

Ahora mapeamos los datos priorizando intereses y limpiamos los datos eliminando columnas no utilizadas.

\begin{figure}[H]
    \centering
    \includegraphics[width=0.8\textwidth]{18.png}
    \caption{Limpieza de datos.}
\end{figure}

\subsection{Construcción del árbol de decisión}
Probamos el top 1 de los diferentes datos mapeados:

\begin{figure}[H]
    \centering
    \includegraphics[width=0.8\textwidth]{19.png}
    \caption{Construccion de tablas con su top 1.}
\end{figure}

\begin{itemize}
    \item Mood.
    \begin{figure}[H]
        \centering
        \includegraphics[width=0.8\textwidth]{20.png}
        \caption{Tabla de mood con top 1.}

    \end{figure}
    \item Artista.
    \begin{figure}[H]
        \centering
        \includegraphics[width=0.8\textwidth]{21.png}
        \caption{Tabla de artista con top 1.}

    \end{figure}
    \item Género.
    \begin{figure}[H]
        \centering
        \includegraphics[width=0.8\textwidth]{22.png}
        \caption{Tabla de genero con top 1.}

    \end{figure}
    \item Tempo.
    \begin{figure}[H]
        \centering
        \includegraphics[width=0.8\textwidth]{23.png}
        \caption{Tabla de tempo con top 1.}

    \end{figure}
    \item Duración.
    \begin{figure}[H]
        \centering
        \includegraphics[width=0.8\textwidth]{24.png}
        \caption{Tabla de duración con top 1.}

    \end{figure}
    \item Edad.
    \begin{figure}[H]
        \centering
        \includegraphics[width=0.8\textwidth]{25.png}
        \caption{Tabla de edad con top 1.}

    \end{figure}
\end{itemize}

Finalmente, creamos el árbol de decisión con \texttt{sklearn}, definiendo los parámetros necesarios. Realizamos pruebas para determinar la profundidad ideal del árbol:

\begin{figure}[H]
    \centering
    \includegraphics[width=0.8\textwidth]{26.png}
    \caption{Construccion de arbol de decision.}
\end{figure}

\begin{figure}[H]
    \centering
    \includegraphics[width=0.8\textwidth]{27.png}
    \caption{Pruebas.}
\end{figure}

\begin{figure}[H]
    \centering
    \includegraphics[width=0.8\textwidth]{28.png}
    \caption{Tabla para visualizar el valor optimo del Max Depth.}
\end{figure}

\begin{verbatim}
Profundidad óptima encontrada: 7.
\end{verbatim}

Visualizamos el árbol configurado y verificamos su precisión.

\begin{figure}[H]
    \centering
    \includegraphics[width=0.8\textwidth]{29.png}
    \caption{Verificamos la precision.}
\end{figure}

\begin{figure}[H]
    \centering
    \includegraphics[width=0.8\textwidth]{../tree1.png}
    \caption{Arbol de decision.}
\end{figure}

\begin{figure}[H]
    \centering
    \includegraphics[width=0.8\textwidth]{30.png}
    \caption{Precision del arbol.}
\end{figure}

\section{Resultados}
Para probar el árbol de decisión, evaluamos dos canciones: \textit{Havana} de Camila Cabello (top) y \textit{Believer} de Imagine Dragons (no top):

\begin{itemize}
    \item \textbf{Havana – Camila Cabello}: Resultado del modelo.
    \begin{figure}[H]
        \centering
        \includegraphics[width=0.8\textwidth]{31.png}
        \caption{Codificacion para analizar la cancion.}

    \end{figure}

    \begin{figure}[H]
        \centering
        \includegraphics[width=0.8\textwidth]{32.png}
        \caption{Resultado.}

    \end{figure}

    \item \textbf{Believer – Imagine Dragons}: Resultado del modelo.
    \begin{figure}[H]
        \centering
        \includegraphics[width=0.8\textwidth]{33.png}
        \caption{Codificacion para analizar la cancion.}

    \end{figure}

    \begin{figure}[H]
        \centering
        \includegraphics[width=0.8\textwidth]{34.png}
        \caption{Resultado.}

    \end{figure}
\end{itemize}

\texttt{El modelo predice que la canción \textit{Havana} de Camila Cabello es un top 1 con una probabilidad de 90.1\% y que \textit{Believer} de Imagine Dragons no es un top 1 con una probabilidad de 65\%.}

\section{Conclusión}
Se presentaron inconvenientes con la codificación debido a errores en el código original y un valor incorrecto de \texttt{max\_depth} (4 en lugar de 7). Se instalaron las paqueterías necesarias y se solucionó un problema con Graphviz que no identificaba el \texttt{PATH}. A pesar de estos desafíos, modificando y limpiando los datos, logramos obtener un resultado favorable. De cualquier manera durante esta actividad se aprendió a usar el algoritmo de árboles de decisión, su construcción y la importancia de la limpieza de datos.

\section{Referencias}
\begin{itemize}
    \item Materiales de clase (2025). UANL.
    \item Bagnato J (2019). \textit{Aprende Machine Learning}. Leanpub.
\end{itemize}

\end{document}
